In diesem Versuch betrachten wir das Verhalten von Licht im Wasser. Als wir das von der Halogenlampe ausgesendete Licht mit einem Polarisationsfilter betrachten stellen wir fest, dass das gestreute Licht in unterschiedliche Richtungen mit unterschiedlichen Polarisationen gestreut wird. Zusätzlich stellen wir fest, dass die Intensitätsmaxima der unterschiedlichen Polarisationen immer senkrecht auf dem originalen Lichtstrahl stehen. Dieses Verhalten ist anhand der Wassermoleküle erklärbar, die angeregt von der Halogenlampe licht entsenden. Die verschiedenen Polarisationen entstehen durch die unterschiedlich ausgerichteten Wassermoleküle. Die Maxima, da diese Richtung die vom Molekül bevorzugte Richtung ist.