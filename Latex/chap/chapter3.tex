\section{Beobachtungen}
In diesem Versuchsteil werden zwischen den Polarisator und den Analysator Glimmerplaettchen und auf Diatraeger aufgebrachte Klebefimlmbildchen gebracht.
Bei eingebrachten glimmerplaettchen oder Klebefimlmbildchen ist bei Rotation eine Farbaenderung zu beobachten.
An den Farbmustern ist klar der Aufbau des Glimmerplaettchens oder das Klebemuster der Klebefimlmbildchen zu erkennen.
Bei den Klebefimlmbildchen ist die Anzahl der Lagen anhand der Farbuebergaenge erkennbar.
Es ist weiterhin erkennbar, dass das Glimmerplaettchen deutlich staerkere farbuebergaenge zeigt, was zu vermuten Laesst, dass die Doppelbrechung oder die frequenzabhaengigkeit der Doppelbrechung groesser ist als bei Tesafilm \cite[S 548]{HECHT5}.
Bei dem Glimmerplaettchen in \ref{fig:glimmerplaettchen_A} und \ref{glimmerplaettchen_B} ist sehr schoen zu erkennen, dass bei einer Rotation um 90 Grad sich die beobachtbaren Farben gerade umkehren.
\begin{figure}
	\includegraphics[width=\textwidth]{../Data/Bilder/DSC00224.JPG}
	\label{fig:klebe_chaos}
	\caption{zufaellig uebereinandergelegte Klebestreifen}
\end{figure}
\begin{figure}
	\includegraphics[width=\textwidth]{../Data/Bilder/DSC00227.JPG}
	\label{fig:klebe_stern_A}
	\caption{Sternfoermig aufgelegte klebefimstreifen in Analysatorstellung A}
\end{figure}
\begin{figure}
	\includegraphics[width=\textwidth]{../Data/Bilder/DSC00229.JPG}
	\label{fig:klebe_stern_B}
	\caption{Sternfoermig aufgelegte Klebefilmstreifen in Analysatorstellung B}
\end{figure}
\begin{figure}
	\includegraphics[width=\textwidth]{../Data/Bilder/DSC00234.JPG}
	\label{fig:glimmerplaettchen_A}
	\caption{Glimmerplaettchen in Analysatorstellung A}
\end{figure}
\begin{figure}
	\includegraphics[width=\textwidth]{../Data/Bilder/DSC00236.JPG}
	\label{fig:glimmerplaettchen_B}
	\caption{Glimmerplaettchen in Analysatorstellung B}
\end{figure}
\section{Interpretation}
Sowohl das Glimmerplaettchen als auch der Tesafilm zeigen Doppelbrechungseigenschaften.
Diese Doppelbrechungseigenschaften ist aufgrund der fundamentalen eigenschaften der Doppelbrechung frequenzabhaengig \cite[S. 548]{HECHT5}.


