\section{Beobachtungen}
In diesem Versuchsteil werden Plexiglasteile eingespannt und mithilfe von Polarisiertem Licht in dem Bekannten aufbau von Polarisator, Probe und Analysator bei Belastung untersucht.
\begin{figure}
	\includegraphics[width=\textwidth]{../Data/Bilder/DSC00259.JPG}
	\caption{Kraftfrei eingespanntes T-Stueck bei gleicher Stellung von Polarisator und analysator}
	\label{fig:TF}
\end{figure}
\begin{figure}
	\includegraphics[width=\textwidth]{../Data/Bilder/DSC00259.JPG}
	\caption{Tstueck ohne Last bei $\phi_a$ - $\phi_p$ = $\pi/2$}
	\label{fig:TOL}
\end{figure}
\begin{figure}
	\includegraphics[width=\textwidth]{../Data/Bilder/DSC00261.JPG}
	\caption{Tstueck unter Last bei $\phi_a$ - $\phi_p$ = $\pi/2$}
	\label{fig:TL}
\end{figure}
\begin{figure}
	\includegraphics[width=\textwidth]{../Data/Bilder/DSC00252.JPG}
	\caption{Tstueck mit Bohrungen an den Ecken unter Last bei $\phi_a$ - $\phi_p$ = $\pi/2$. Man kann schoen erkennen, wie die Bohrung an den Ecken Die bei \ref{fig:TL} punktuell anliegende Spannung im Bauteil verteilt}
	\label{fig:TLL}
\end{figure}
Bei Kraftfreier Einspannung der verschiedenen Proben ist erkennbar, dass das Plexiglas keine bemerkenswerten Doppelbrechungseigenschaften hat~\ref{fig:TF}.
Bei einem Winkelunterschied von 90 Grad zwischen Polarisator und Analysator sind die Kanten des Plexiglases schwach zu erkennen~\ref{fig:TOL}. Sonst bleibt das Innere des T-Stuecks schwarz.
Sobald man allerdings beginnt mithilfe der Spannschraube Kraft auf ein eingespanntes Plexiglasstueck auszuueben werden die Bereiche des Plexiglases Heller~\ref{fig:TL}, was eine Rotation der Polarisation bedeutet.
Diese Stammt dann von belastungsinduzierten Doppelbrechungseigenschaften des Plexiglases.
Dies heisst wiederum, dass sich anhand der Helligket der verschiedenen Bereiche die Belastung des Plexiglases untersuchen laesst.
Die belasteten bereiche zeigen im Gegensatz zu den Glimmer und Tesafilbildchen keine verschiedenen Farben, was vermuten laesst, das die Doppelbrechung zumindest im sichtbaren Bereich einigermassen Frequenzunabhaengig ist.
Die Ecken der Plexiglasteile bei weitem die hellsten Punkte der verschiedenen Teststuecke.
Dies ist durchgehend der Fall.
\section{Interpretation}
Wie bereits gesagt, laesst die Rotation der Polarisationsrichtung eine Doppelbrechungseigenschaften im Plexiglas vermuten.
Bei Spannung bilden sich vermutlich Dipole entlang des Kraftflusses, dies haette zur Folge, dass Plexiglas entlang der Kraftlinien einen anderen Bechungsindex hat.
Dies fuehrt dann dazu, dass die Polarisationsrichtung verdreht wird.
Da das Plexiglas von durchgehend gleicher Dicke ist, sind kaum Dickenabhaengige Polarisations-winkelaenderungen zu erwarten.
Man kann an den Verschiedenen Proben gut erkennen, das sich an Ecken hohe Spannungen aufbauen \ref{fig:TL}.
Sich diese allerdings bei abgerundeten oeffnungen oder kanten entlang der gesamten Kante verteilen und damit das Material insgesamt weniger belasten \ref{fig:TLL}.
Diese Beobachtung kann zum Beispiel die Rundungen an Flugzeugfenstern und den zunehmend organischen Stil vieler Maschinen (z.B. Autos) erklaeren.
