In diesem Versuch untersuchen wir die Polarisation einiger Lichtstrahlen.
\section{Lineare Polarisation}

Zunächst untersuchen wir weißes Licht, welches wir mit einem Polfilter linear polarisieren. Aus den Messwerten ergibt sich der erste Graph.
\begin{figure}
	\centering
	\includegraphics[width=0.8\textwidth]{../Data/Linear_1.png}
	\caption{Messwerte des weißen, linear polarisierten Lichts}
\end{figure}
Es fällt auf, dass selbst beim Minimum der Messwerte der Wert 0 nicht erreicht wird. Dies liegt daran, dass der verwendete Polarisationsfilter nicht alle Wellenlängen gleich polarisiert. Um bessere Messwerte zu erhalten verwenden wir deshalb einen Interferenzfilter der nur die Wellenlänge $ \lambda=630\; nm $ durchlässt. Mit diesem Versuchsaufbau ergibt sich der zweite Graph.
\begin{figure}
	\centering
	\includegraphics[width=0.8\textwidth]{../Data/Linear_2.png}
	\caption{Messwerte des gefilterten, linear polarisierten Lichts}
\end{figure}

Hierbei ist erkennbar, dass durch den Filter eine bessere Polarisation des Lichts entsteht.
\section{Elliptische Polarisation}
Im zweiten Versuchsteil erzeugen wir mithilfe von Glimmplättchen eine elliptische Polarisation. Es wurde darauf geachtet, dass die Amplituden der Strahlen im Glimmer gleich groß ist. Zunächst führen wir den Versuch mit dem Plättchen der Dicke 60$ \;\mu m $ durch, wodurch sich Graph drei ergibt.
\begin{figure}
	\centering
	\includegraphics[width=0.8\textwidth]{../Data/Elliptisch_1.png}
	\caption{Messwerte des elliptisch polarisierten Lichts mit 60$ \;\mu m$ Glimmplättchen}
\end{figure}
bei der zweiten Messung ist scheinbar ein Fehler aufgetreten, da wir hier einen besseren Kreis messen, als bei der zirkularen Polarisation.
\begin{figure}
	\centering
	\includegraphics[width=0.8\textwidth]{../Data/Elliptisch_2.png}
	\caption{Messwerte des elliptisch polarisierten Lichts mit 50$ \;\mu m$ Glimmplättchen}
\end{figure}
Wir vermuten, dass dies mit dem Versuchsaufbau zusammenhängt, da wir nach dem Versuch bemerkten, dass es einen großen Einfluss auf die Messwerte hat ob die Hand nach dem verstellen des Polarisationsfilters auf diesem bleibt oder heruntergenommen wird. 
\section{Zirkulare Polarisation}
In diesem Versuchsteil erzeugen wir mithilfe eines $\lambda/4  $ -Plättchens eine zirkulare Polarisation.
\begin{figure}
	\centering
	\includegraphics[width=0.8\textwidth]{../Data/Zirkular.png}
	\caption{Messwerte des zirkular polarisierten Lichts}
\end{figure}
Die somit gemessenen Messwerte entsprechen jedoch eher einer elliptischen Polarisation. Der Grund hierfür könnte ein ungenaues einstellen des $\lambda/4  $ -Plättchens sein.
