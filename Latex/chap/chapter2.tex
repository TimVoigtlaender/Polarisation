In dieser Aufgabe berechnen wir die Aktivität des Cs Präparates.
Hierbei gilt die Formel:
\begin{align*}
A=\dfrac{N-N_{grund}}{(1-p_{tot})\cdot p_{nw}}
\end{align*}
Wobei $ N_{grund}=8242 $ die Anzahl an Messwerten ist, die durch das Untergrundspektrum hinzukommt.
\begin{center}
	Tabelle 1: Messwerte zu Aufgabe 2
\end{center}
\begin{center}
	\begin{tabular}{c|c|c|c|c}
		Abstand in cm & Totzeit $p_{tot}$ in \% & Nachweisw. $p_{nw}$ in \% & Messwerte N & Aktivität in Bq \\ \hline
		1       &          24,24          &            5,0            &   995460    &      86872      \\
		3       &           8,9           &            1,2            &   353885    &     105392      \\
		6       &           3,4           &            0,4            &   143751    &     116899
	\end{tabular} 
\end{center}

Man würde hierbei erwarten, dass die Aktivitäten unabhängig vom Abstand konstant bleiben. Dies ist hier nicht der Fall, da scheinbar ein weiterer Abstands-abhängiger Faktor hinzukommt. Wir halten es für wahrscheinlich, dass die Tabelle zur Nachweiswahrscheinlichkeit nicht optimal auf den Versuch abgestimmt ist, weswegen der Abstand ungewollter weise einen Einfluss auf die Aktivität hat.