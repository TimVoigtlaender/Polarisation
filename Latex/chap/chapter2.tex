In diesem Versuchsteil berechnen wir die Differenz der Brechungsindizes der Glimmplättchen aus versuch 1. Hierzu nutzen wir das Verhältnis zwischen größtem und kleinstem Radius. Wie man bereits in Aufgabe 1 sehen kann sind die entstehenden Graphen Rotationssymmetrisch, weswegen es ausreicht das Verhältnis zwischen kleinstem und größtem Messwert zu betrachten. Aus diesem Verhältnis $ \frac{b}{a} $ kann mit $ \frac{b^2}{a^2}=\frac{r_{min}}{r_{max}} $ und $ \Delta\phi=2\cdot \arctan\left( \frac{b}{a}\right)  $ die Phasendifferenz $ \Delta\phi $ berechnen. Zusammen mit 
\begin{align*}
\Delta\text{n}=\frac{\lambda\cdot\Delta\phi}{2\pi\cdot d}
\end{align*}
ergibt sich somit:
\begin{align*}
\Delta\text{n}=\frac{\lambda\cdot \arctan\left( \sqrt{\frac{r_{min}}{r_{max}}}\right)}{\pi\cdot d}
\end{align*}
Hierbei ist d die Dicke des Plättchens und $ r_{min}/r_{max} $ der kleinste bzw. größte Messwert. $ \lambda $ ist aufgrund des verwendeten Filters als 630$ \; $nm annehmbar.
\begin{figure}
	\centering
	\includegraphics[width=0.8\textwidth]{../Daten/Aufgabe_2.png}
	\caption{Skizze zur Veranschaulichung der Rechnung [Quelle: Musterprotokoll]}
\end{figure}
Somit können wir nun die gesuchte Brechzahldifferenz berechnen.
\begin{center}
	Tabelle 1: Brechzahldifferenzen der gemessenen Plättchen
\end{center}
\begin{center}
	\begin{tabular}{|c|c|c|}
		\hline
		Dicke d  & 60$\;\mu$m & 50$\;\mu$m \\ \hline
		r$_{min}$ & 8,75$\;$mV & 8,05$\;$mV \\ \hline
		r$_{max}$ & 11,6$\;$mV & 9,15$\;$mV \\ \hline
		$\Delta$n &   0,0024   &   0,003    \\ \hline
	\end{tabular} 
\end{center}
Diese Werte passen zwar recht gut zueinander, vergleicht man sie jedoch mit den Ergebnissen aus dem Musterprotokoll($ \Delta n_{muster}\approx0,15 $) stellt man fest, dass die von uns errechneten Werte um etwa zwei Größenordnungen zu klein sind. Wir vermuten, dass dies an den bereits in Versuch 1 erwähnten Problemen an der Messapertur liegt.